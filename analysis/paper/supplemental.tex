% Options for packages loaded elsewhere
\PassOptionsToPackage{unicode}{hyperref}
\PassOptionsToPackage{hyphens}{url}
%
\documentclass[
]{article}
\usepackage{lmodern}
\usepackage{amssymb,amsmath}
\usepackage{ifxetex,ifluatex}
\ifnum 0\ifxetex 1\fi\ifluatex 1\fi=0 % if pdftex
  \usepackage[T1]{fontenc}
  \usepackage[utf8]{inputenc}
  \usepackage{textcomp} % provide euro and other symbols
\else % if luatex or xetex
  \usepackage{unicode-math}
  \defaultfontfeatures{Scale=MatchLowercase}
  \defaultfontfeatures[\rmfamily]{Ligatures=TeX,Scale=1}
\fi
% Use upquote if available, for straight quotes in verbatim environments
\IfFileExists{upquote.sty}{\usepackage{upquote}}{}
\IfFileExists{microtype.sty}{% use microtype if available
  \usepackage[]{microtype}
  \UseMicrotypeSet[protrusion]{basicmath} % disable protrusion for tt fonts
}{}
\makeatletter
\@ifundefined{KOMAClassName}{% if non-KOMA class
  \IfFileExists{parskip.sty}{%
    \usepackage{parskip}
  }{% else
    \setlength{\parindent}{0pt}
    \setlength{\parskip}{6pt plus 2pt minus 1pt}}
}{% if KOMA class
  \KOMAoptions{parskip=half}}
\makeatother
\usepackage{xcolor}
\IfFileExists{xurl.sty}{\usepackage{xurl}}{} % add URL line breaks if available
\IfFileExists{bookmark.sty}{\usepackage{bookmark}}{\usepackage{hyperref}}
\hypersetup{
  pdftitle={Supplemental methods},
  hidelinks,
  pdfcreator={LaTeX via pandoc}}
\urlstyle{same} % disable monospaced font for URLs
\usepackage[margin=1in]{geometry}
\usepackage{graphicx,grffile}
\makeatletter
\def\maxwidth{\ifdim\Gin@nat@width>\linewidth\linewidth\else\Gin@nat@width\fi}
\def\maxheight{\ifdim\Gin@nat@height>\textheight\textheight\else\Gin@nat@height\fi}
\makeatother
% Scale images if necessary, so that they will not overflow the page
% margins by default, and it is still possible to overwrite the defaults
% using explicit options in \includegraphics[width, height, ...]{}
\setkeys{Gin}{width=\maxwidth,height=\maxheight,keepaspectratio}
% Set default figure placement to htbp
\makeatletter
\def\fps@figure{htbp}
\makeatother
\setlength{\emergencystretch}{3em} % prevent overfull lines
\providecommand{\tightlist}{%
  \setlength{\itemsep}{0pt}\setlength{\parskip}{0pt}}
\setcounter{secnumdepth}{-\maxdimen} % remove section numbering

\title{Supplemental methods}
\author{}
\date{\vspace{-2.5em}}

\begin{document}
\maketitle

Animal handling methods, biologger specifications, and calculation of
arrival and departure dates are described in Robinson, et
al.~\textsuperscript{1}. Satellite tracking data were filtered and
processed using the R package crawl\textsuperscript{2,3} to eliminate
inaccurate location points and interpolate between locations. The
resulting latitude and longitude estimates were used to calculate great
circle distance (in kilometers) from the Año Nuevo breeding beach
(37.1083°N, 122.3366°W) for each time-latitude-longitude point in the
MATLAB function \texttt{distance()}. Across all seals, foraging trip
timing (mean ± SD day-of-year) was as follows: departure 157 ± 9,
turnaround 287 ± 40, and arrival 15 ± 8 (Figure 1C). Therefore, outbound
trip durations were 130 ± 41 days, and inbound trip durations were 93 ±
41 days. Turnaround dates were calculated using Gaussian kernels with
standard deviation 6 hours using custom functions in R. Code and data
for a subset of animals are available on GitHub (link available upon
review of manuscript) (NOTE: The GitHub repo will be archived on Zenodo,
so cite that instead when ready.) Drift rate dates were calculated using
a custom MATLAB code based on kernel density estimation of fine-scale
changes in depth over time (drift rate, measured in
meters/sec).\textsuperscript{4} Dates are presented as day-of-year
relative to parturition date, with negative numbers indicating dates
before pupping. All analyses were carried out in R v4.0.2. A linear
mixed-effects model of turnaround date (relative to pupping date) as a
function of turnaround distance and buoyancy change date was run in the
package lme4\textsuperscript{5} after scaling and centering the
continuous variables and including individual as a random effect.

Figure S1 will go here. There's a bug in
analysis/data/9SupplementalFigure.R:105.

\hypertarget{refs}{}
\leavevmode\hypertarget{ref-robinson2012}{}%
1. Robinson, P., Costa, D., Crocker, D., Gallo-Reynoso, J., Champagne,
C., Fowler, M., Goetsch, C., Goetz, K., Hassrick, J., Hückstädt, L., et
al. (2012). Foraging behavior and success of a mesopelagic predator in
the northeast pacific ocean: Insights from a data-rich species, the
northern elephant seal. PLoS ONE \emph{7}, e36728.

\leavevmode\hypertarget{ref-johnson2008}{}%
2. Johnson, D., London, J., Lea, M., and Durban, J. (2008).
Continuous-time correlated random walk model for animal telemetry data.
Ecology \emph{89}, 1208--1215.

\leavevmode\hypertarget{ref-johnson2016}{}%
3. Johnson, D., Josh M. London (NOAA), and Kenady (2016). Crawl: V2.0.

\leavevmode\hypertarget{ref-robinson2010}{}%
4. Robinson, P., Simmons, S., Crocker, D., and Costa, D. (2010).
Measurements of foraging success in a highly pelagic marine predator,
the northern elephant seal. Journal of Animal Ecology \emph{79},
1146--1156.

\leavevmode\hypertarget{ref-bates2015}{}%
5. Bates, D., Mächler, M., Bolker, B., and Walker, S. (2015). Fitting
linear mixed-effects models usinglme4. Journal of Statistical Software
\emph{67}.

\end{document}
